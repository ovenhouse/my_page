\documentclass {article}

\usepackage {amsthm}
\usepackage {amssymb}
\usepackage {amsmath}
\usepackage {fancyhdr}

\usepackage[margin=1.15in]{geometry}

\newenvironment{prob}[2][]{\begin{trivlist}
\item[\hskip \labelsep {\bfseries #1}\hskip \labelsep {\bfseries #2.}]}{\end{trivlist}}

\newcommand{\softreturn}{\ \\ \relax}

\thispagestyle {fancy}
\lhead {MTH 103}
\chead {Quiz 1 Solutions}
\rhead {Section 65}

\begin {document}

\softreturn
Make sure to \textbf{clearly show all your work}. Grades will be based on your intermediate steps
as well as the final answer. Unless the problem says otherwise, give \textbf{exact answers} (not decimal approximations).

\vspace {1cm}

\begin {prob}{1} (4 points)
    Solve the following linear equation, and write your solution(s) \textbf{using set notation}.
    \[ 12x - (x+5) = 4(6-x) + 1 \]
\end {prob}

\vspace {6cm}

\begin {prob}{2} (4 points)
    You have two different kinds of lemonade. One is $8\%$ sugar and one is $15\%$ sugar.
    The $8\%$ lemonade tastes too watered down, and the $15\%$ lemonade is too sweet, and so you 
    decide to mix the two together. How much of each should you put into the mixture in order to
    end up with 2 liters of lemonade with $11\%$ sugar? (\textbf{Write answers in decimal form, rounded to three places})
\end {prob}

\vspace {6cm}

\softreturn
\textbf{Amount (in liters) of 8\% lemonade}: \underline{\hspace{6cm}} \\ \\
\textbf{Amount (in liters) of 15\% lemonade}: \underline{\hspace{6cm}}

\newpage

\begin {prob}{3} (4 points)
    Solve the following rational equation and write your solution(s) \textbf{using set notation}.
    \[ \frac{1}{x-1} + \frac{2}{x+1} = \frac{3}{x^2-1} \]
\end {prob}

\vspace {6cm}

\begin {prob}{4} (4 points)
    Solve the following equation for the variable $A$:
    \[ B = \frac{A+B+C}{A-B} \]
\end {prob}

\vspace {6cm}

\begin {prob}{5}
    Answer the following questions from the syllabus: \\ \\
    \begin {tabular}{cll}
        $(a)$ & What kind of calculator is recommended for this course? & \underline{\hspace{4cm}} \\ [2ex]
        $(b)$ & What day will we take the first exam?                   & \underline{\hspace{4cm}} \\ [2ex]
        $(c)$ & What is the last day to drop this class with a refund?  & \underline{\hspace{4cm}} \\ [2ex]
        $(d)$ & What sections of the book will be covered on Exam 2?    & \underline{\hspace{4cm}}
    \end {tabular}
\end {prob}

\newpage

\section* {Solutions}

\vspace {1cm}

\begin {prob}{1}
    \begin {align*}
        12x - (x+5) &= 4(6-x) + 1 \\
        12x - x - 5 &= 4 \cdot 6 - 4x + 1 \tag{distribute} \\
        11x - 5     &= 25 - 4x \tag{simplify both sides} \\
        15x         &= 30 \tag{add $4x$ and add 5} \\
        x           &= 2 \tag{divide by 15}
    \end {align*}
    The solution set is $\{2\}$.
\end {prob}

\vspace {1cm}

\begin {prob}{2}
    First of all, let's call the lemonade that is $8\%$ sugar ``lemonade $A$'', and let's call the lemonade
    that is $15\%$ sugar ``lemonade $B$''. We are going to pour some of lemonade $A$ and some of lemonade $B$ 
    into the same pitcher, and we want to end up with 2 liters of lemonade which is $11\%$ sugar.
    The problem is to determine how much (meaning volume, in liters) of each type of lemonade we need to use.
    The equation we will use to describe this situation
    will describe the amount of sugar. It will be:
    \[ \textnormal{``sugar from $A$''} + \textnormal{``sugar from $B$''} = \textnormal{``sugar from 11\% mixture''} \]
    This just says that if we add up the amount of sugar from each type of lemonade, we should get the amount of sugar
    in our final mixture. Let's let $x$ be the volume (in liters) of lemonade $A$ that we will need. Since lemonade $A$
    is 8\% sugar, $0.08x$ is the volume of sugar (in liters) represented by ``sugar from $A$'' in the equation. Since
    we want to end up with 2 liters of the final mixture, and we are already using $x$ liters of lemonade $A$, then
    we must use $2-x$ liters of lemonade $B$. So $0.15(2-x)$ is the volume of sugar represented by
    ``sugar from $B$'' in the equation. Finally, we want to end up with 2 liters of lemonade, which is 11\% sugar,
    so $0.11(2 \mathrm{L}) = 0.22 \mathrm{L}$ is the amount of sugar represented by ``sugar from 11\% mixture'' on
    the right side of the equation. So our equation to solve is:
    \begin {align*}
        0.08x + 0.15(2-x)   &= 0.11(2) \\
        0.08x + 0.3 - 0.15x &= 0.22  \tag{distribute} \\
        0.08x - 0.15x       &= 0.22 - 0.3 \tag{subtract 0.3} \\
        -0.07x              &= -0.08 \tag{simplify} \\
        x                   &= \frac{-0.08}{-0.07} = \frac{8}{7} \tag{divide} \\
        x                   &\approx 1.143
    \end {align*}
    We have concluded that we need to use 1.143 liters of lemonade $A$. So we must use $2-x$ liters of lemonade $B$.
    Using the value of $x$ we have just found, we see that we need $2-1.143 = 0.857$ liters of lemonade $B$. \\
    This problem is similar to \textbf{Examples 6,7} from \textbf{Section 1.2} of the lecture notes,
    and problems \textbf{9,10}, and \textbf{11} from the example homework problems of \textbf{Section 1.2}.
\end {prob}

\newpage

\begin {prob}{3}
    \begin {align*}
        \frac{1}{x-1} + \frac{2}{x+1} &= \frac{3}{x^2-1} \\
        \frac{1}{x-1} + \frac{2}{x+1} &= \frac{3}{(x-1)(x+1)} \tag{factor denominator} \\
        &\phantom{=} \\
        \frac{(x-1)(x+1)}{x-1} + \frac{2(x-1)(x+1)}{x+1} &= \frac{3(x-1)(x+1)}{(x-1)(x+1)} \tag{mult. by LCD:$(x-1)(x+1)$} \\
        &\phantom{=} \\
        (x+1) + 2(x-1) &= 3 \tag{cancel} \\
        x + 1 + 2x - 2 &= 3 \tag{distribute} \\
        3x - 1 &= 3 \tag{simplify} \\
        3x &= 4 \tag{add 1} \\
        x &= \frac{4}{3} \tag{divide}
    \end {align*}
    This solution doesn't violate any of the restrictions (the restrictions are that $x \neq 1$ and $x \neq -1$),
    so the solution set is just $\left\{ \frac{4}{3} \right\}$. \\
    For other examples like \#3, see \textbf{Examples 6,7,8} from \textbf{Section 1.1} of the lecture notes,
    and problems \textbf{6,8} from the example homework problems of \textbf{Section 1.1}.
\end {prob}

\vspace {1cm}

\begin {prob}{4}
    \begin {align*}
        B &= \frac{A+B+C}{A-B} \\
        B(A-B) &= A+B+C \tag{mult. by A-B} \\
        B \cdot A - B \cdot B &= A+B+C \tag{distribute} \\
        BA - B^2 &= A+B+C \\
        BA - A   &= B^2 + B + C \tag{collect A terms} \\
        (B-1)A   &= B^2 + B + C \tag{factor} \\
        A        &= \frac{B^2 + B + C}{B - 1} \tag{divide}
    \end {align*}
    For other examples like \#4, see \textbf{Examples 12,14} from \textbf{Section 1.1} of the lecture notes,
    and problems \textbf{11} and \textbf{13} from the example homework problems at the end of \textbf{Section 1.1}
\end {prob}

\vspace {1cm}

\begin {prob}{5} \ \\ \\ \relax
    \begin {tabular}{cll}
        $(a)$ & What kind of calculator is recommended for this course? & \underline{TI 83/84} \\ [2ex]
        $(b)$ & What day will we take the first exam?                   & \underline{January 30} \\ [2ex]
        $(c)$ & What is the last day to drop this class with a refund?  & \underline{January 31} \\ [2ex]
        $(d)$ & What sections of the book will be covered on Exam 2?    & \underline{1.7 - 3.1}
    \end {tabular}
\end {prob}

\end {document}
