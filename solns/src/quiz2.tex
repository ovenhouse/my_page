\documentclass {article}

\usepackage {amsthm}
\usepackage {amssymb}
\usepackage {amsmath}
\usepackage {fancyhdr}

\usepackage[margin=1.15in]{geometry}

\newenvironment{prob}[2][]{\begin{trivlist}
\item[\hskip \labelsep {\bfseries #1}\hskip \labelsep {\bfseries #2.}]}{\end{trivlist}}

\newcommand {\fspace}{\vspace {\fill}}
\newcommand {\blank} [1] {\underline{\hspace{#1cm}}}

\thispagestyle {fancy}
\lhead {\textbf{Name:}}
\chead {MTH 103 \\ Section 65}
\rhead {Quiz 2 \\ 1/23/14}

\begin {document}

\ \\ \\ \relax
\noindent Make sure to \textbf{clearly show all your work}. Grades will be based on your intermediate steps
as well as the final answer. Unless the problem says otherwise, give \textbf{exact answers} (not decimal approximations).

\vspace {1cm}

\begin {prob}{1} (5 points)
    Solve the following quadratic equation, and write your solution(s) \textbf{using set notation}.
    \[ (2x-1)^2 + 3 = 18 \]
\end {prob}

\fspace

\begin {prob}{2} (5 points)
    Solve the following rational equation, and write your solution(s) \textbf{using set notation}.
    \[ \frac{3}{x-1} + \frac{7}{x+5} = 2 \]
\end {prob}

\fspace

\newpage

\begin {prob}{3} (4 points)
    You throw a tennis ball up in the air from 1 meter off the ground. The height of the ball after $t$ seconds
    is given by
    \[ h = -4.9t^2 + 4t + 1 \]
    How long will it take for the ball to hit the ground?
\end {prob}

\fspace

\begin {prob}{4} (2 points)
    Write down a quadratic equation whose solution set is $\{1, 3\}$.
\end {prob}

\fspace

\begin {prob}{5} (4 points)
    Determine the number of real solutions to each of the following quadratic equations. You do
    not need to find the solutions...just tell how many there are. \textbf{(Hint: compute the discriminant)} \\ \\ \\
    \begin {tabular}{cll}
        $(a)$ & $x^2 - 5x + 6 = 0$  & \blank{3} \\
        & & \\
        $(b)$ & $2x^2 + 3 = 4x$     & \blank{3} \\
        & & \\
        $(c)$ & $x^2 + 4x = 3$      & \blank{3} \\
        & & \\
        $(d)$ & $(x-8)^2 = 0$       & \blank{3} \\
    \end {tabular}
\end {prob}

\newpage

\section* {Solutions}

\begin {prob}{1}
    \[ (2x-1)^2 + 3 = 18 \]
    This problem is particularly well-suited for the square root method:
    \begin {align*}
        (2x-1)^2 + 3 &= 18 \\
        (2x - 1)^2 &= 15 \tag{subtract 3} \\
        2x - 1 &= \pm \sqrt{15} \tag{take square root} \\
        2x &= 1 \pm \sqrt{15} \tag{add 1} \\
        x &= \frac{1 \pm \sqrt{15}}{2} \tag{divide}
    \end {align*}
\end {prob}
The solution set can be written either as $\left\{ \frac{1 \pm \sqrt{15}}{2} \right\}$ or
$\left\{ \frac{1+\sqrt{15}}{2}, \frac{1-\sqrt{15}}{2} \right\}$.
\\ \\
Alternatively, you could've used the quadratic formula. You just need to expand everything
and get zero on one side of the equation first:
\begin {align*}
    (2x-1)^2 + 3 &= 18 \\
    4x^2 - 4x + 1 + 3 &= 18 \tag{expand} \\
    4x^2 - 4x + 4 &= 18 \tag{simplify} \\
    4x^2 - 4x - 14 &= 0 \tag{subtract 18} \\
    2x^2 - 2x - 7 &= 0 \tag{divide by 2} \\
    x &= \frac{2 \pm \sqrt{(-2)^2 - 4(2)(-7)}}{2(2)} \tag{use quad. form.} \\
    x &= \frac{2 \pm \sqrt{60}}{4} \\
    x &= \frac{2 \pm \sqrt{4 \cdot 15}}{4} \\
    x &= \frac{2 \pm 2 \sqrt{15}}{4} \\
    x &= \frac{2 \left( 1 \pm \sqrt{15}\right)}{4} \\
    x &= \frac{1 \pm \sqrt{15}}{2}
\end {align*}
This problem was similar to \textbf{Examples 4,5} from the \textbf{Section 1.3} notes
and \textbf{Problems 11,12} from the \textbf{Section 1.3} homework exercises.

\newpage

\begin {prob}{2}
    \[ \frac{3}{x-1} + \frac{7}{x+5} = 2 \]
\end {prob}
We need to clear the fractions, and the LCD we need is $(x-1)(x+5)$.
So we'll multiply both sides of the equation by that, and we'll get
a quadratic to solve:
\begin {align*}
    \frac{3}{x-1} + \frac{7}{x+5} &= 2 \\
    \frac{3(x-1)(x+5)}{x-1} + \frac{7(x-1)(x+5)}{x+5} &= 2(x-1)(x+5) \\
    3(x+5) + 7(x-1) &= 2(x-1)(x+5) \\
    3x + 15 + 7x - 7 &= 2(x^2 + 4x - 5) \\
    10x + 8 &= 2x^2 + 8x - 10 \\
    0 &= 2x^2 - 2x - 18 \\
    0 &= x^2 - x - 9 \\
    x &= \frac{1 \pm \sqrt{(-1)^2 - 4(1)(-9)}}{2} \\
    x &= \frac{1 \pm \sqrt{37}}{2}
\end {align*}
So the solution set is $\left\{ \frac{1 \pm \sqrt{37}}{2} \right\}$.\\ \\
This problem is similar to \textbf{Examples 1,2} from the \textbf{Section 1.4} notes
and \textbf{Problems 1,2} from the \textbf{Section 1.4} homework exercises.

\vspace {1cm}

\begin {prob}{3}
    We need to find when the ball hits the ground, and so we want the height to be zero.
    So we set $h = 0$ and solve:
    \begin {align*}
        0 = -4.9t^2 + 4t + 1 \\
        t &= \frac{-4 \pm \sqrt{4^2 - 4(-4.9)(1)}}{2(-4.9)} \\
        t &= \frac{-4 \pm \sqrt{35.6}}{-9.8}
    \end {align*}
    One of these answers is negative, and one is positive. We want the positive one
    (the one with the minus sign in this case), which is 1.017 seconds. \\ \\
    This problem is similar to \textbf{Example 7} from \textbf{Section 1.4} notes
    and \textbf{Problems 7,8} from \textbf{Section 1.4} homework exercises.
\end {prob}

\newpage

\begin {prob}{4}
    We just need to multiply $x-1$ and $x-3$ together and set it equal to zero:
    \begin {align*}
        (x-1)(x-3) &= 0 \\
        x^2 - 4x + 3 &= 0
    \end {align*}
    Also, if you multiply this equation by any real number $r$, you will get another
    equation with the same solutions:
    \[ rx^2 - 4rx + 3r = 0 \]
    This problem is similar to \textbf{Example 6} from \textbf{Section 1.4} notes
    and \textbf{Problem 6} from the \textbf{Section 1.4} homework exercises.
\end {prob}

\vspace {1cm}

\begin {prob}{5} \ \\ \\ \relax
$(a)$ $x^2 - 5x + 6 = 0$ The discriminant is $(-5)^2 - 4(1)(6) = 25 - 24 = 1 > 0$. So there are 2 real solutions.
\\ \\
$(b)$ $2x^2 - 4x + 3 = 0$ The discriminant is $(-4)^2 - 4(2)(3) = 16 - 24 = -8 < 0$. So there are no real solutions.
\\ \\
$(c)$ $x^2 + 4x - 3 = 0$ The discriminant is $4^2 - 4(1)(-3) = 16 + 12 = 28 > 0$. So there are 2 real solutions.
\\ \\
$(d)$ $(x-8)^2 = 0$ After expanding, this is equivalent to the equation
$x^2 - 16x + 64 = 0$. The discriminant is $(-16)^2 - 4(1)(64) = 0$. So there is 1 real solution.
\end {prob}
This problem is similar to \textbf{Example 9} from \textbf{Section 1.3} notes and
\textbf{Problems 15,16,17} from \textbf{Section 1.3} homework exercises.

\end {document}
